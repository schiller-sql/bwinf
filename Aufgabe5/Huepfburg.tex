\documentclass[a4paper,10pt,ngerman]{scrartcl}
\usepackage{babel}
\usepackage[T1]{fontenc}
\usepackage[utf8x]{inputenc}
\usepackage[a4paper,margin=2.5cm,footskip=0.5cm]{geometry}

% Die nächsten vier Felder bitte anpassen:
\newcommand{\Aufgabe}{Aufgabe 5: Hüpfburg}        % Aufgabennummer und Aufgabennamen angeben
\newcommand{\TeamId}{?????}                       % Team-ID aus dem PMS angeben
\newcommand{\TeamName}{????}                      % Team-Namen angeben
\newcommand{\Namen}{Tony Borchert, Lennart Protte}% Namen der Bearbeiter/-innen dieser Aufgabe angeben

% Kopf- und Fußzeilen
\usepackage{scrlayer-scrpage, lastpage}
\setkomafont{pageheadfoot}{\large\textrm}
\lohead{\Aufgabe}
\rohead{Team-ID: \TeamId}
\cfoot*{\thepage{}/\pageref{LastPage}}

% Position des Titels
\usepackage{titling}
\setlength{\droptitle}{-1.0cm}

% Für mathematische Befehle und Symbole
\usepackage{amsmath}
\usepackage{amssymb}

% Für Bilder
\usepackage{graphicx}

% Für Algorithmen
\usepackage{algpseudocode}

% Für Quelltext
\usepackage{listings}
\usepackage{color}
\definecolor{mygreen}{rgb}{0,0.6,0}
\definecolor{mygray}{rgb}{0.5,0.5,0.5}
\definecolor{mymauve}{rgb}{0.58,0,0.82}
\lstset{
    keywordstyle=\color{blue},commentstyle=\color{mygreen},
    stringstyle=\color{mymauve},rulecolor=\color{black},
    basicstyle=\footnotesize\ttfamily,numberstyle=\tiny\color{mygray},
    captionpos=b, % sets the caption-position to bottom
    keepspaces=true, % keeps spaces in text
    numbers=left, numbersep=5pt, showspaces=false,showstringspaces=true,
    showtabs=false, stepnumber=2, tabsize=2, title=\lstname
}
\lstdefinelanguage{JavaScript}{ % JavaScript ist als einzige Sprache noch nicht vordefiniert
    keywords={break, case, catch, continue, debugger, default, delete, do, else, finally, for, function, if, in, instanceof, new, return, switch, this, throw, try, typeof, var, void, while, with},
    morecomment=[l]{//},
    morecomment=[s]{/*}{*/},
    morestring=[b]',
    morestring=[b]",
    sensitive=true
}

% Diese beiden Pakete müssen zuletzt geladen werden
\usepackage{hyperref} % Anklickbare Links im Dokument
\usepackage{cleveref}

% Daten für die Titelseite
\title{\textbf{\Huge\Aufgabe}}
\author{\LARGE Team-ID: \LARGE \TeamId \\\\
\LARGE Team-Name: \LARGE \TeamName \\\\
\LARGE Bearbeiter/-innen dieser Aufgabe: \\
\LARGE \Namen\\\\}
\date{\LARGE\today}

\begin{document}

    \maketitle
    \tableofcontents

    \vspace{0.5cm}

    \section{Lösungsidee}\label{sec:losungsidee}
    Der Aufgabenstellung zu entnehmen ist,
    dass sich beide Spieler immer genau gleichzeitig,
    um genau einen Knoten,
    entlang der Kantenrichtung bewegen und der Parcours als absolviert gilt,
    wenn sich beide gleichzeitig zum gleichen Knoten bewegen.
    Dabei ist die Startposition immer Identisch.
    Als Datenstruktur für den Parcours eignet sich ein gerichteter Graph.
    Die Pfeile zwischen den Kästchen werden durch gerichtete Kanten repräsentiert,
    während die Knoten repräsentativ für die Felder stehen.
    Die Problemstellung der Aufgabe ist es,
    einen Knoten in einem gerichteten Graphen zu finden,
    welcher von zwei unterschiedlichen Ausgangsknoten,
    mit der gleichen Anzahl an Schritten zu erreichen ist.
    Dabei darf die Schrittfolge nie entgegengesetzt zur Kantenrichtung verlaufen.
    Dies kann erreicht werden,
    indem für beide Spieler zeitgleich eine Breitensuche schrittweise durchgeführt wird.
    Diese enthält keine Heuristiken und ist damit eine uninformierte Suche.
    Wenn es sich um einen lösbaren Parcours handelt,
    wird für beide Spieler im selben Schritt mindestens ein gleicher Knoten gefunden.
    Sollte der Parcours nicht lösbar sein, %Diesen Bandwurmsatz entknoten :D
    wird für beide Spieler gleichzeitig,
    die Menge,
    der zu einem Zeitpunkt markierten Knoten,
    der Menge,
    der zu einem vorherigen Zeitpunkt markierten Knoten,
    entsprechen.
    Dies lässt sich beweisen, indem ...


    \section{Umsetzung}\label{sec:umsetzung}


    \section{Beispiele}\label{sec:beispiele}


    \section{Quellcode}\label{sec:quellcode}\label{LastPage}

\end{document}
