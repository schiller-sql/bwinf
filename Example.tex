\documentclass[a4paper,10pt,ngerman]{scrartcl}
\usepackage{babel}
\usepackage[T1]{fontenc}
\usepackage[utf8x]{inputenc}
\usepackage[a4paper,margin=2.5cm,footskip=0.5cm]{geometry}

% Die nächsten vier Felder bitte anpassen:
\newcommand{\Aufgabe}{Aufgabe 1: \LaTeX-Dokument} % Aufgabennummer und Aufgabennamen angeben
\newcommand{\TeamId}{?????}                       % Team-ID aus dem PMS angeben
\newcommand{\TeamName}{Team-Name}                 % Team-Namen angeben
\newcommand{\Namen}{Vor- und Nachnamen}           % Namen der Bearbeiter/-innen dieser Aufgabe angeben

% Kopf- und Fußzeilen
\usepackage{scrlayer-scrpage, lastpage}
\setkomafont{pageheadfoot}{\large\textrm}
\lohead{\Aufgabe}
\rohead{Team-ID: \TeamId}
\cfoot*{\thepage{}/\pageref{LastPage}}

% Position des Titels
\usepackage{titling}
\setlength{\droptitle}{-1.0cm}

% Für mathematische Befehle und Symbole
\usepackage{amsmath}
\usepackage{amssymb}

% Für Bilder
\usepackage{graphicx}

% Für Algorithmen
\usepackage{algpseudocode}

% Für Quelltext
\usepackage{listings}
\usepackage{color}
\definecolor{mygreen}{rgb}{0,0.6,0}
\definecolor{mygray}{rgb}{0.5,0.5,0.5}
\definecolor{mymauve}{rgb}{0.58,0,0.82}
\lstset{
    keywordstyle=\color{blue},commentstyle=\color{mygreen},
    stringstyle=\color{mymauve},rulecolor=\color{black},
    basicstyle=\footnotesize\ttfamily,numberstyle=\tiny\color{mygray},
    captionpos=b, % sets the caption-position to bottom
    keepspaces=true, % keeps spaces in text
    numbers=left, numbersep=5pt, showspaces=false,showstringspaces=true,
    showtabs=false, stepnumber=2, tabsize=2, title=\lstname
}
\lstdefinelanguage{JavaScript}{ % JavaScript ist als einzige Sprache noch nicht vordefiniert
    keywords={break, case, catch, continue, debugger, default, delete, do, else, finally, for, function, if, in, instanceof, new, return, switch, this, throw, try, typeof, var, void, while, with},
    morecomment=[l]{//},
    morecomment=[s]{/*}{*/},
    morestring=[b]',
    morestring=[b]",
    sensitive=true
}

% Diese beiden Pakete müssen zuletzt geladen werden
%\usepackage{hyperref} % Anklickbare Links im Dokument
\usepackage{cleveref}

% Daten für die Titelseite
\title{\textbf{\Huge\Aufgabe}}
\author{\LARGE Team-ID: \LARGE \TeamId \\\\
\LARGE Team-Name: \LARGE \TeamName \\\\
\LARGE Bearbeiter/-innen dieser Aufgabe: \\
\LARGE \Namen\\\\}
\date{\LARGE\today}

\begin{document}

    \maketitle
    \tableofcontents

    \vspace{0.5cm}

    \textbf{Anleitung:} Trage oben in den Zeilen 8 bis 11 die Aufgabennummer, die Team-ID, den Team-Namen und alle Bearbeiter/-innen dieser Aufgabe mit Vor- und Nachnamen ein.
    Vergiss nicht, auch den Aufgabennamen anzupassen (statt "`\LaTeX-Dokument"')!

    Dann kannst du dieses Dokument mit deiner \LaTeX-Umgebung übersetzen.

    Die Texte, die hier bereits stehen, geben ein paar Hinweise zur Einsendung.
    Du solltest sie aber in deiner Einsendung wieder entfernen!

    \section{Lösungsidee}\label{sec:losungsidee}
    Die Idee der Lösung sollte hieraus vollkommen ersichtlich werden, ohne dass auf die eigentliche Implementierung Bezug genommen wird.
    Hier sollte eben die Idee der L¨osung beschrieben werden.
    Dazu geh¨ort meistens die Abstraktion des Problems, die Idee selbst und die Argumentation, wieso die Idee richtig ist.
    Der erste Teil der L¨osungsidee sollte ublicherweise darin bestehen, das gegebene Problem ¨
    formalisieren, Mehrdeutigkeiten entscheiden zu formalisieren.
    Dazu geh¨ort auch, dass eventuelle Uneindeutigkeiten in der Aufgabenstellung entschieden und klar dokumentiert werden.
    Das Ergebnis sollte eine formale, mathematische Darstellung des Problems sein – so einfach wie m¨oglich und nicht unn¨otig kompliziert.
    Es geht nicht darum, irgendjemanden davon zu uberzeugen, dass das Pro- ¨ blem schwierig ist.
    Vielmehr gl¨anzen gute L¨osungen dadurch, dass sie das Problem erstaunlich einfach darstellen.
    Gute, veranschaulichende Beispiele sind naturlich erlaubt ¨ und erwunscht.
    Es ist einerseits wichtig, die Algorithmen in Worten so zu beschreiben, dass der Leser
    die richtige Intuition und Vorstellung bekommt.
    Andererseits sollten die Beschreibungen so pr¨azise sein, dass der Leser den Algorithmus ohne viel nachzudenken selbst implementieren kann.
    Fur eine pr ¨ ¨azise Beschreibung eignet sich Pseudocode besonders gut.
    Triviale Schritte im Algorithmus mussen nat ¨ urlich nicht als Pseudocode ausgef ¨ uhrt wer- ¨ den.
    Wie immer gilt: weniger ist mehr.
    Echter Quelltext oder Implementationsdetails
    haben in der L¨osungsidee nichts zu suchen.
    Es kann interessant sein, alternative L¨osungsverfahren zu kurz zu beschreiben und zu
    erkl¨aren, weshalb diese nicht verwendet worden sind.
    Wenn dies von der Aufgabenstellung impliziert ist, kann es auch n¨otig sein, mehrere L¨osungsverfahren zu implementieren
    und diese zu vergleichen.
    In einem solchen Fall ist es sehr wichtig, die Vergleichsergebnisse ubersichtlich im Kapitel „Programm-Ablaufprotokolle“ zu pr¨asentieren
    ( zum Beispiel in einer Tabelle)und zu interpretieren.
    Wenn die vergleichsrelevanten Ergebnisse auf mehreren Seiten verteilt sind,
    wird sich kaum ein Bewerter die Muhe machen und diese ¨ zusammensuchen.
    Nachdem die L¨osungsidee beschrieben wurde, muss in den allermeisten F¨allen eine theoretische Analyse folgen.
    Oft sollte hier die Laufzeitkomplexit¨at des Algorithmus genannt und begrundet werden.
    ¨ 12 Auch die Speicherkomplexit¨at ist interessant, falls diese nicht trivial ist.
    Wenn nachgewiesen werden kann, dass das vorliegende Problem NPvollst¨andig ist, ist dies meistens Extrapunkte wert.
    Die Analyse sollte knapp und richtig
    sein: es ist nicht hilfreich, wenn die Anzahl der Zyklen auf einer selbst-definierten Rechnerarchitektur auf zehn Seiten genau berechnet wird.
    Vielmehr sollten eine oder mehrere
    Variablen definiert werden, die die Gr¨oße der Eingabe beschreiben.
    Mithilfe dieser Variablen kann die Laufzeit- bzw.
    Speicherkomplexit¨at dann in O-Notation13 angegeben
    werden.
    Wenn man von der Laufzeit einer ”
    durchschnittlichen Eingabe“ sprechen m¨ochte
    (dieser Begriff muss naturlich definiert sein) und die Analyse dieser Laufzeit zu kompli- ¨
    ziert ist, kann man mithilfe von Zeitmessungen eine Hypothese aufstellen (diese sollte
    selbstverst¨andlich auch als reine Hypothese dargestellt werden).
    In der ersten Runde sind solche Messungen selten sinnvoll.
    Nicht nur in der theoretischen Analyse ist es erlaubt und sinnvoll, mathematische
    Symbole wie P oder Q
    zu verwenden, wenn ein Zusammenhang nicht einfacher dargestellt werden kann – Ausdrucke mit diesen Symbolen sind meistens leichter zu verstehen ¨
    als andere Konstruktionen oder umst¨andliche S¨atze, die den Zusammenhang beschreiben.
    Oft k¨onnen (und sollten) kompliziertere Ausdrucke mit einem einfachen Beispiel ¨
    erl¨autert werden.

    \section{Umsetzung}\label{sec:umsetzung}
    Hier wird kurz erläutert, wie die Lösungsidee im Programm tatsächlich umgesetzt wurde.
    Hier können auch Implementierungsdetails erwähnt werden.
    Nachdem unter „L¨osungsidee“ das Problem theoretisch gel¨ost wurde, muss die L¨osung nun durch ein Programm realisiert werden.
    Diese Transformation von Idee ins Programm soll hier vermittelt werden.
    Beim Verst¨andnis kann Pseudocode oder zum Teil auch ein UML-Diagramm enorm helfen.
    In der Umsetzung geht es darum, den Bezug zwischen der L¨osungsidee und der ImpleL¨osungsidee und
    Implementation mentation herzustellen.
    Nur selten sollte hier der tats¨achliche Quelltext abgedruckt sein.
    Auf gar keinen Fall soll der Code Zeile fur Zeile erkl ¨ ¨art werden.
    Wenn der Quelltext nicht selbst-erkl¨arend ist (und damit ist nicht eine Flut von Kommentaren gemeint!), ist er
    oft in schlechtem Zustand.
    In den allermeisten F¨allen reicht es anzugeben, in welchen
    Methoden die Algorithmen aus der L¨osungsidee implementiert sind.
    Dies ist der richtige Abschnitt, um darauf einzugehen, wie bestimmte Konzepte der
    L¨osungsidee elegant im Programm modelliert wurden (in diesem Fall durfen nat ¨ urlich ¨
    auch kurze Ausschnitte des Quelltextes gezeigt werden).
    Wenn euer Programm (ohne Benutzeroberfl¨ache) l¨anger als wenige hundert Zeilen ist,
    kann es fur den Bewerter hilfreich sein, ein UML Klassendiagramm ¨
    14 oder ein Flussdiagramm15 zu sehen.
    Solche Diagramme haben allerdings nur ihren Platz in der Dokumentation, wenn sie zum Verst¨andnis beitragen und sich auf relevanten Code beziehen. Es
    wird kein Klassendiagramm ben¨otigt, was zeigt, welches die privaten Felder eurer C++
    Klasse sind (Das k¨onnen die Bewerter auch im Quelltext nachlesen.), und auch keines,
    das darstellt, dass euer Hauptfenster 5 Buttons enth¨alt.
    Keinesfalls sollte hier eine "Programm-Dokumentation“ im Sinne eines Benutzer handbuches geschrieben werden.
    Es sollte davon ausgegangen werden, dass m¨ogliche
    Leser bzw.
    Bediener sowohl mit dem Programmieren als auch mit der konkreten Aufgabenstellung vertraut sind.
    Ist daruberhinaus eine Benutzerdokumentation notwendig, ¨
    so ist die Benutzeroberfl¨ache offenbar verbesserbar.

    \section{Beispiele}\label{sec:beispiele}
    Genügend Beispiele einbinden!
    Die Beispiele von der BwInf-Webseite sollten hier diskutiert werden,
    aber auch eigene Beispiele sind sehr gut – besonders wenn sie Spezialfälle abdecken.
    Aber bitte nicht 30 Seiten Programmausgabe hier einfügen!
    Hier soll dokumentiert werden, wie das Programm sich unter verschiedenen Eingaben verh¨alt.
    Die Eingaben sind am besten so gew¨ahlt, sodass der Leser
    nach diesem Abschnitt uberzeugt ist, dass das Programm korrekt funktioniert und ¨
    die relevanten F¨alle l¨osen kann (z.B.fur ¨ N = 1000000).
    Zu jeder L¨osung geh¨oren 3 bis 5 Beispiele (Programmeingaben und dazugeh¨orige -
    ausgaben oder Zwischenschritte), aus denen ersichtlich wird, wie das Programm sich
    in unterschiedlichen Situationen verh¨alt.
    Auch wenn 3 Beispieleingaben in der Aufgabe
    vorgegeben sind, sollte man zus¨atzlich noch 3 Eingaben selbst entwerfen, dann erkl¨aren,
    wieso man was wie gew¨ahlt hat, und das Ergebnis kommentieren.
    Mit der Wahl der Eingaben sollte man m¨oglichst viele F¨alle abdecken (solche, die etwa im allt¨aglichen Betrieb
    zu erwarten sind, aber auch solche, die Randwerte oder Sonderf¨alle darstellen, und das
    Programm vielleicht an seine Grenzen bringen).
    Schw¨achen Schw¨achen des Programms sollte man kommentieren. (Das wird positiv gewertet,
    kommentieren das Weglassen solcher Kommentare negativ!).
    Bringt das Programm etwa nur bei jedem zehnten Aufruf ein gutes Ergebnis, so sollte man dokumentieren, dass man das
    Programm fur die Beispiele zehnmal hat laufen lassen, oder eben gleich das Programm ¨
    ¨andern, so dass es den Algorithmus zehnmal ausfuhrt (das geh ¨ ¨ort dann aber naturlich ¨
    schon entsprechend in der L¨osungsidee dokumentiert).
    Die ” Protokolle ggf.
    Ablaufprotokolle“ sollen sinnvoll darstellen, was bei einem Programmablauf
    kurzen ¨ passiert.
    Handelt es sich bei der Aufgabe etwa um ein Computerspiel, bei dem 1000
    Zuge gemacht werden, so ist eine Liste der Z ¨ uge sicherlich nicht sinnvoll. (Die Einsendung ¨
    wird von Menschen gelesen, nicht von Computern!) Stattdessen sollten vielleicht einzelne
    Stellungen mit jeweils n¨achsten paar Zugen ausgegeben werden, anhand derer dann ¨
    dokumentiert wird, wie sich die implementierte Methode stategisch auswirkt.
    Beim Drucken von Ablaufprotokollen kann man umweltfreundlich sein und Geld
    Hintergrund sparen, wenn man Screenshots von schwarzen Konsolenfenstern invertiert. Das sollte
    man daher tun, es ¨andert nichts an der Sache.

    \section{Quellcode}\label{sec:quellcode}
    Unwichtige Teile des Programms sollen hier nicht abgedruckt werden.
    Dieser Teil sollte nicht mehr als 2–3 Seiten umfassen, maximal 10.
    Variablen- oder Funktionsnamen haben nichts in der L¨osungsidee zu suchen!
    Die L¨oder Umsetzungsungsidee soll abstrakt die L¨osung beschreiben.
    Das Programm ist lediglich eine m¨ogliche Umsetzung der Idee.
    Dies zwingt zum einen dazu, dass die Idee eigenst¨andig formuliert sein muss, sich also
    nicht hinter Aussagen wie „XY wird dann in der Funktion xy() berechnet“ verstecken kann.
    Das hilft beim Verst¨andnis, denn normalerweise wird man die L¨osung von vorne
    nach hinten lesen, und wie xy() funktioniert, ist zu dem Zeitpunkt, an dem dieser Satz
    gelesen wird, weder bekannt noch relevant.
    Zum anderen ist der Name der Funktion(xy()) nur von sehr geringem Interesse:
    Ist die L¨osungsidee richtig und arbeitet das Programm korrekt (das sind Dinge die den Bewerter interessieren),
    so ist wohl auch xy() (oder wie auch immer die Funktion heißen mag) korrekt.
    Der relevante Teil des Quelltextes muss in ausgedruckter Form vorliegen. (Auf den Daunrelevant tentr¨ager gehen wir sp¨ater ein.) Die Unterscheidung zwischen relevantem und unrelevantem Code entscheidet sich an der Frage, was in der Aufgabenstellung gefragt ist. In
    unserem Beispiel war die Eingabe der Spielsituation, die wir textbasiert uber die Kon- ¨
    sole realisiert haben, nicht relevant.
    Daher haben wir ihren Code in separate Dateien
    ausgelagert und diese nicht abgedruckt. D
    iese Trennung unterscheidet aber nicht nur,
    Modular was abgedruckt wird, sondern ist auch gutes Softwaredesign.
    Wir haben ein Interface (zu
    programmieren deutsch: Schnittstelle) entwickelt, die den L¨osungsalgorithmus von seiner Verwendung
    isoliert.
    Das Interface selber, in drehzahl strategy.h definiert,
    \begin{lstlisting}[label={lst:lstlisting}]
    1 #include ”cards.h”
    2
    3 // Computes the expected final score fora given hand
    4 double ExpectedScore ( const Cards& hand ) ;
    5
    6 // Computes what cards to play next , given a hand and the sum of rolled dice .
    7 // Returns an empty set of cards if no move is possible .
    8 Cards BestNextMove ( const Cards& hand , int roll ) ;
    \end{lstlisting}
    haben wir nicht abgedruckt, weil es schon in der Umsetzung beschrieben und einfach aus
    der bereits abgedruckten drehzahl strategy.cpp ableitbar war.
    Solche Interfaces machen
    das gesamte Programm testbarer, einfacher und wartbarer, und helfen sowohl beim L¨osen
    des Problems, da man ein großes Problem in zwei kleine unterteilt, die man unabh¨angig
    voneinander l¨osen kann, als auch sp¨ater beim Kommunizieren bzw.
    Dokumentieren der
    eigenen L¨osung, da alle so isolierten Komponenten unabh¨angig voneinander sind —
    sofern die Interfaces exakt spezifiziert sind16
    .
    Sind die Ein- und Ausgabe, die graphische Oberfl¨ache (zum Beispiel Fensterklassen),
    Ereignisbehandlungsmethoden, das Lesen und Schreiben von Dateien, oder ¨ahnliche
    Komponenten des Programms nicht ein zentraler Teil der Aufgabenstellung, so w¨are
    es auch falsch, sie zu einem zentralen Teil der L¨osung zu machen: Dieser Code ist oft
    recht technisch und an das verwendete System gekoppelt, zum Beispiel an Windowing
    Toolkits oder Runtime Libraries. Tragen diese Teile des Codes nichts zur L¨osung des
    eigentlichen Problemes bei, so sind sie nicht relevant.
    Aus dieser Erkenntnis folgt, dass solche Teile des Codes auch keinen Wert fur die ¨
    L¨osung des Problems tragen. Man muss sie somit nicht nur nicht abdrucken, sondern
    eigentlich auch gar nicht schreiben: Mit der Konsoleeingabe haben wir in unserem Fall
    das Problem der Eingabe m¨oglichst einfach gel¨ost: Eine graphische Oberfl¨ache (englisch:
    Konzentration auf graphical user interface, GUI) h¨atte keinen Mehrwert geschaffen, sondern lediglich den
    das Wesentliche nicht-relevanten Codeteil verl¨angert.
    Schreibt man doch GUI-Code, so sollte man ihn, soweit es geht, vom Rest des Programms isolieren.
    Dazu bietet sich das ”
    Model-View-Controller-Pattern“ an17 MVC : Es darf
    als bekannt vorausgesetzt werden, und die konsistente und konsequente Verwendung der
    Begriffe ”
    Model“, ”
    View“ und ”
    Controller“ im Code und in der Dokumentation helfen
    nach der Erw¨ahnung des Patterns wiederum enorm beim Verst¨andnis.



\end{document}
